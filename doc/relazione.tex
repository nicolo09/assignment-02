\documentclass[11pt,notitlepage]{article}
\usepackage[a4paper, margin={2cm, 2.2cm}]{geometry}
\usepackage[a-2b]{pdfx}[2018/12/22]
\usepackage{alltt, fancyvrb, url}
\usepackage{graphicx}
\usepackage[utf8]{inputenc}
\usepackage{float}
\usepackage{hyperref}
\usepackage{subcaption}
\usepackage{numprint}
\usepackage{slashbox}
\usepackage{pgfplots}
\usepackage{amsmath}
\usepackage{lipsum}
\usepackage{tikz}
\pgfplotsset{compat=1.9,
            tick label style={font=\tiny},
            label style={font=\tiny}}

% Questo commentalo se vuoi scrivere in inglese.
\usepackage[italian]{babel}

\usepackage[italian]{cleveref}

\title{Programmazione concorrente e distribuita \\ Assignment 02}

\author{
    \href{mailto:nicolo.guerra@studio.unibo.it}{Nicolò Guerra, matricola 0001179571} \\
    \href{mailto:filippo.casadei9@studio.unibo.it}{Filippo Casadei, matricola 0001179572} \\
    \href{mailto:emma.leonardi2@studio.unibo.it}{Emma Leonardi, matricola 0001193227}
    }
\date{\today}

\begin{document}
\maketitle
\renewcommand{\thesection}{\arabic{section}}
\section{Analisi del problema}

L'assignment riguarda la creazione di due artefatti differenti: una libreria per la ricerca asincrona di dipendenze in classi, package e progetti Java e una applicazione
reattiva per la visualizzazione delle dipendenze di progetti Java.

\section{Versione asincrona}
//TODO: scrivere meglio parte di analisi del problema
Il problema principale della versione asincrona riguarda l'analisi dei file \texttt{.java}, in particolare è importante che le letture da file non blocchino il sistema e che vengano
eseguite concorrentemente al flusso del programma per evitare di bloccare il sistema.

Per la libreria è stato utilizzato il framework \texttt{Vertx} che offre supporto alla creazione di sistemi asincroni. In particolare la libreria è formata da un oggetto 
\texttt{DependencyAnalyzer} che si occupa di analizzare i file \texttt{.java} e di restituire le dipendenze in modo asincrono. Fornisce tre metodi principali:
\begin{itemize}
    \item \texttt{getClassDependencies}: cerca le dipendenze di una classe
    \item \texttt{getPackageDependencies}: cerca le dipendenze che le classi di un package hanno verso tipi esterni ad esso
    \item \texttt{getProjectDependencies}: cerca le dipendenze che le classi di un progetto hanno verso tipi esterni ad esso
\end{itemize}
Ogni metodo ritorna un Future generico sul tipo di report che il metodo produce

\section{Conclusioni}


\end{document}
