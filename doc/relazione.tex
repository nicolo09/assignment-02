\documentclass[11pt,notitlepage]{article}
\usepackage[a4paper, margin={2cm, 2.2cm}]{geometry}
\usepackage[a-2b]{pdfx}[2018/12/22]
\usepackage{alltt, fancyvrb, url}
\usepackage{graphicx}
\usepackage[utf8]{inputenc}
\usepackage{float}
\usepackage{hyperref}
\usepackage{subcaption}
\usepackage{numprint}
\usepackage{slashbox}
\usepackage{pgfplots}
\usepackage{amsmath}
\usepackage{lipsum}
\usepackage{tikz}
\pgfplotsset{compat=1.9,
            tick label style={font=\tiny},
            label style={font=\tiny}}

% Questo commentalo se vuoi scrivere in inglese.
\usepackage[italian]{babel}

\usepackage[italian]{cleveref}

\title{Programmazione concorrente e distribuita \\ Assignment 02}

\author{
    \href{mailto:nicolo.guerra@studio.unibo.it}{Nicolò Guerra, matricola 0001179571} \\
    \href{mailto:filippo.casadei9@studio.unibo.it}{Filippo Casadei, matricola 0001179572} \\
    \href{mailto:emma.leonardi2@studio.unibo.it}{Emma Leonardi, matricola 0001193227}
    }
\date{\today}

\begin{document}
\maketitle
\renewcommand{\thesection}{\arabic{section}}
\section{Analisi del problema}

L'assignment riguarda la creazione di due artefatti differenti: una libreria per la ricerca asincrona di dipendenze in classi, package e progetti Java e una applicazione
reattiva per la visualizzazione delle dipendenze di progetti Java.

\section{Versione asincrona}
%//TODO: scrivere meglio parte di analisi del problema
Il problema principale della versione asincrona riguarda l'accesso in lettura ai file \texttt{.java}, in particolare è importante che le letture da file non blocchino 
il sistema e che vengano eseguite concorrentemente per evitare di bloccare il flusso principale del programma.

Per l'implementazione della libreria è stato utilizzato il framework \texttt{Vertx} che offre supporto alla creazione di sistemi asincroni. In particolare la libreria è formata 
da un oggetto \texttt{DependencyAnalyzer} che si occupa di analizzare i file \texttt{.java} e di restituire le dipendenze in modo asincrono. Fornisce tre metodi principali:
\begin{itemize}
    \item \texttt{getClassDependencies}: cerca le dipendenze di una classe
    \item \texttt{getPackageDependencies}: cerca le dipendenze che le classi di un package hanno verso tipi esterni ad esso
    \item \texttt{getProjectDependencies}: cerca le dipendenze che le classi di un progetto hanno verso tipi esterni ad esso
\end{itemize}
Ogni metodo ritorna un Future generico sul tipo di report che il metodo produce.

La ricerca per le classi è stata implementata utilizzando i metodi di accesso al filesystem messi a disposizione dall'oggetto \texttt{FileSystem} di Vertx, che 
permette di eseguire operazioni non bloccanti, e concatenando a queste le operazioni di parsing e analisi dei file.

Trattandosi di una libreria non sappiamo a priori il contesto all'interno del quale verrà eseguito il codice, per questo abbiamo deciso di non effettuare il deploy di
un oggetto \texttt{Verticle} ma di lasciare la scelta all'utente. Questo da flessibilità alla libreria ma non permette di sapere a priori quale istanza di Vertx verrà utilizzata.
Per questo abbiamo lasciato la possibilità di passare un oggetto \texttt{Vertx} al costruttore dell'oggetto, nel caso in cui questo non venga fatto e venga utilizzato il costruttore
di default, si cerca di recuperare l'oggetto \texttt{Vertx} corrente tramite il metodo \texttt{Vertx.currentContext()} e infine se non esiste un contesto corrente si crea un nuovo
oggetto con chiamata a \texttt{Vertx.vertx()}.

Per i package e i progetti è stato necessario implementare un metodo di ricerca ricorsiva che, partendo dalla directory principale del progetto, analizza i file \texttt{.java}
raccogliendo con un \texttt{Future.all} i risultati asincroni dell'analisi delle classi e dei sotto-package.

Per evitare di fornire risultati parziali e quindi incorretti qualunque problema durante l'analisi di un file \texttt{.java} causa il fallimento dell'intera operazione 
e quindi del relativo \texttt{Future}.

\subsection{Utilizzo versione asincrona}
Per provare la versione asincrona è necessario avviare il metodo main della classe \texttt{pcd.ass02.async.VertxApp} con i parametri:
\begin{itemize}
    \item \texttt{--class [class file path]}: per analizzare una singola classe
    \item \texttt{--package [package folder path]}: per analizzare un package
    \item \texttt{--project [project folder path]}: per analizzare un progetto
\end{itemize}

\section{Versione reattiva}
Mentre nell'approccio asincrono il main si preoccupa di mettere in coda delle task da eseguire e di gestire il loro completamento (o fallimento) in qualche modo, 
l'approccio reattivo si differenzia nel fatto che si ha un componente osservabile che continua a eseguire in background e il flusso principale del programma deve 
occuparsi di reagire in qualche modo agli eventi che questo osservabile fa scattare. 

Deve inoltre, essendo i nostri osservabili infiniti, occuparsi di farlo terminare al momento opportuno. L'analisi delle dipendenze in questo caso è perciò affidata 
all'esecuzione di un \texttt{Observable} di tipo cold che istanzia un nuovo thread, questo si occupa al suo avvio di cercare le dipendenze in background nei file 
e inviare un nuovo oggetto dependency ogni volta che ne trova una, una volta esplorati tutti i file il thread si mette in osservazione della cartella di progetto
selezionata per eventuali cambiamenti e in caso di modifiche ai sorgenti li rianalizza e notifica la view del cambiamento che si occupa di aggiornare il grafo, 
permettendo così di reagire in tempo reale a modifiche dei file. 

La ricerca avviene chiamando sull'oggetto \texttt{ReactiveDependencyFinder} il metodo \\
\texttt{findAllClassesDependencies} che restituisce un \texttt{Observable} che emette un oggetto \\
\texttt{ClassDependencyInfo} ogni volta che trova una dipendenza o che una dipendenza già notificata varia.

Il controller mantiene un grafo delle dipendenze come oggetto \texttt{DependenciesGraph} e la view lo disegna (tramite un oggetto wrapper per compatibilità con la libreria grafica)
ogni volta che il controller riceve una notifica di cambiamento dall'observable.

La versione reattiva mostra inoltre le classi totali analizzate e il numero di dipendenze trovate, si può inoltre fermare l'analisi reattiva con un apposito pulsante che causa la
terminazione del thread di analisi e la chiusura dell'osservabile.

\subsection{Utilizzo versione reattiva}

Per utilizzare la versione reattiva è necessario avviare il metodo main della classe \\
\texttt{pcd.ass02.reactive.ReactiveApp} e scegliere una cartella di progetto con il tasto \texttt{Select directory}.
A questo punto è possibile avviare l'analisi con il tasto \texttt{Start} e verrà mostrato il grafo delle dipendenze mantenendo l'analisi reattiva fino a quando non si preme 
il tasto \texttt{Stop} o non si chiude il programma.
È inoltre possibile attivare o disattivare l'auto layout del grafo con l'apposita checkbox, effettuare lo zoom utilizzando CTRL+scroll e spostare il grafo con il mouse.

\end{document}
